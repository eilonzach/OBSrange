We present \textit{OBSrange}, a new open-source tool for robustly locating OBS on the seafloor. Acoustic ranging two-way travel-time data between the ship and OBS are inverted for horizontal instrument position, instrument depth, and depth-averaged water sound speed. Our algorithm can account for travel time perturbations due to ship motion between sending and receiving, ray bending through the water column, and a static offset between the GPS and transponder. Uncertainties are calculated for all four parameters using bootstrap resampling, and an F-test grid search provides a 3D confidence ellipsoid around the station. The tool is validated using a synthetic travel-time dataset yielding typical horizontal location errors on the order of $\sim$4~m for 5000~m water depth. Various survey geometries are explored through synthetic tests, and we find that the \textit{PACMAN} survey configuration is most successful at recovering horizontal location, even with an unaccounted for GPS-transponder offset. Optimal survey radius depends on water depth and desired precision ranging from 0.75--1~Nm at 5000~m water depth to $\sim$0.25~Nm at 500~m depth. The \textit{Circle} configuration is unable to resolve depth and water velocity and should be avoided. The \textit{Line} survey pattern, commonly used in short-period OBS deployments, recovers instrument location parallel to the line but has no resolution in the orthogonal direction. If instrument depth and/or water velocity are of particular importance, a survey pattern such as \textit{PACMAN} is desirable, which contains long ship tracks toward and away from the instrument. If GPS-transponder offset is uncertain and cannot be measured, the \textit{Cross2} or \textit{Hourglass} patterns provide the best resolution of depth and water velocity. We apply the tool to the 2018 Young Pacific ORCA deployment yielding an average RMS data misfit of 1.96~ms and revealing a clockwise-rotation pattern in the instrument drifts with a diameter of $\sim$500~km that correlates with a cyclonic mesoscale feature. This observation further demonstrates the precision of \textit{OBSrange} and suggests the possibility of utilizing instrument drift data as an oceanographic tool for estimating depth-integrated flow through the water column.