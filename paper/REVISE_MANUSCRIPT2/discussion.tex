A reliable tool for accurately locating instruments on the seafloor is paramount, given the growing number of ocean bottom deployments. We present the first such open-source OBS locator code that is freely available to the scientific community. One of the primary features of the tool is its ability to provide robust confidence bounds on the 3D instrument position on the seafloor, which will inform recovery cruise efforts as well as provide accurate station metadata, upon which essentially all seismic analyses rely. Furthermore, we have performed the first systematic exploration of survey geometries that we are aware of, which will help streamline future OBS deployments.

At $\sim$5000~m water depth, most survey geometries recover depth and average sound speed velocity equally well. However, the \textit{PACMAN} survey with a radius of $\sim$1~Nm sufficiently recovers all model parameters in the synthetic tests (Figure~\ref{fig:survey_geom_explore}), including $z$ and $V_p$ to within 10~m and 3~m/s, respectively, and horizontal location to within $\sim$3~m. A radius of 0.75~Nm is sufficient for accurate horizontal locations (to within $\sim$4~m) but with increased uncertainty in instrument depth and water velocity. However, the smaller 0.75~Nm radius survey reduces the total survey duration by $\sim$25\% compared to the 1~Nm survey ($\sim$38~min compared to $\sim$50~min for an average ship velocity of 8~kn). If depth and water velocity estimates are of lesser importance and/or time is limited, the smaller 0.75~Nm radius may be desirable. A survey radius larger than 1~Nm is likely not warranted, requiring more ship time at each site for little improvement in misfit. Additionally, failed acoustic returns are more likely to occur at greater distances from the instrument, resulting in data gaps which will negatively impact the inversion. Some ship captains prefer only to steam along straight lines; in such cases, the \textit{Diamond} survey with 1~Nm radius is a viable alternative, given its comparable performance to the \textit{PACMAN} geometry (Figure~\ref{fig:survey_geom_explore}a--c). The radial legs of the survey where the ship travels toward and away from the instrument are crucial for resolving the depth-velocity trade-off. For this reason, the \textit{Circle} configuration cannot independently resolve depth and water velocity and should be avoided.

The \textit{Line} geometry warrants additional discussion as it is commonly used for locating OBS during active-source experiments because it is often the simplest pattern. Parallel to the line, the instrument location is resolved quite well (to within $\sim$4~m). However, the instrument location perpendicular to the line cannot be resolved. This is evident from the resolution matrix as well as the synthetic bootstrap tests. The instrument depth is also poorly resolved with RMS of $\sim$200~m. In order to resolve both horizontal dimensions and depth, an alternative survey geometry with a range of ship-track azimuths (or even two perpendicular lines crossing the instrument, such as the \textit{Cross} or \textit{Hourglass} geometry) may be used.

Optimal survey size scales down with decreasing water depth. Figures~S7--S8 in the electronic supplement show the synthetic tests from Section~\ref{sec:surv_geom_tests} at 2000~m and 500~m. The optimal survey radius shrinks to 0.5~Nm at 2000~m water depth and 0.25~Nm at 500~m depth. Uncertainties decrease with decreasing water depth at the preferred survey radius as well as overall. This decrease in optimal survey size has implications for ray bending corrections in shallow water. Deviations from the straight ray approximation occur most strongly in shallow water at large offsets, especially if there is an abrupt drop in velocity at the thermocline (see Figures~S1, available in the electronic supplement to this article). However, the small optimal survey size at shallow water depth means large offsets are never reached, reducing the importance of ray bending even at shallow depths. For instance, at 0.25~Nm offset for 500~m water depth the perturbation to the travel time is only $\sim$0.06~ms, significantly lower than experimental noise, even with the presence of an abrupt thermocline.

Observations of instrument drift from seasurface to seafloor are byproducts of the location algorithm if instrument drop points are precisely recorded. Figure~\ref{fig:meso_eddy} highlights both the precision of the \textit{OBSrange} algorithm as well as the potential for using instrument drift as an oceanographic observation. A clockwise rotation pattern in instrument drift is observed across the Young Pacific ORCA network that is consistent with a large cyclonic mesoscale feature, providing novel point measurements of depth-integrated flow through the water column that could be used to calibrate models of the vertical shear (Ryan Abernathey, \textit{pers. comm.}). Although there are certainly higher resolution methods of measuring shallow-most characteristics of the water column, such as using an acoustic Doppler current profiler (ADCP), observations tracking from the surface to seafloor may still prove useful. With the further proliferation of seafloor data providing broader spatial and temporal sampling, data such as these could be used to verify models of vertical structure of the full water column. The network-wide depth-averaged water velocity is $\sim$1505~m/s with standard deviation $\sim$4.5~m/s, consistent with the regional decadal average for the month of April ($\sim$1509~m/s) from the 2009 World Ocean Atlas database (see Data and Resources).

Accounting for a relative offset between the shipboard GPS and transponder may be important for correctly resolving depth and average sound speed for some combinations of survey geometry and GPS-transponder offset. The synthetic test in Section~\ref{sec:Comparison_to_previous_tools} shows that if the transponder and GPS are offset by $\sim$14~m and the survey pattern is such that the transponder is systematically positioned further than the GPS from the instrument by $\sim$2.5~m (in 3-dimensions), $z$ may be underestimated by as much as $\sim$28~m. This bias may explain the $\sim$18.6~m shallowing of stations at Young Pacific ORCA compared to depths reported by the shipboard multibeam, where a GPS-transponder offset was not known and no correction was applied. Figure~S9 in the electronic supplement shows results for the same synthetic tests from Section~\ref{sec:surv_geom_tests} without the GPS-transponder correction applied. While the \textit{PACMAN} survey still performs best at recovering horizontal location, it poorly recovers depth and water velocity. However, anti-symmetric patterns (i.e. having both clockwise and counter-clockwise segments and ship tracks toward and away from the instrument) such as \textit{Hourglass} and \textit{Cross2} accurately recover $z$ and $V_p$ by effectively canceling the offset anomaly along the anti-symmetric legs. The specific configuration of the GPS-transponder offset relative to the chosen survey pattern dictates the impact of not correcting the travel times for such an offset. For example, if the GPS and transponder were located at the front and back of the ship, respectively, the circular legs of the survey would be unbiased, with large biases along the radial legs. If the GPS-transponder offset cannot be determined before an experiment and accurate depth and sound speed are desired, an anti-symmetric survey pattern with clockwise/counter-clockwise and radial legs toward/away from the instrument may be used with a slight reduction in horizontal precision.

We find that the Doppler travel-time corrections improve RMS travel-time misfit by only $\sim$0.3~ms ($\sim$7\% reduction) for the synthetic test (Figure~\ref{fig:compare_tool}) and do not improve RMS misfit for the real data. However, the test shows a reduction in horizontal errors of $\sim$2~m ($\sim$40\%) when using the correction, and therefore, we include the Doppler correction as an option in the code. One possible reason why the corrections fail to improve the travel-time misfit for real data may simply be the inability to accurately estimate ship velocity resulting from poor GPS spatial precision and/or poor spatiotemporal sampling along the ship tracks, especially when large data gaps are present. The \textit{PACMAN} survey pattern is quasi-circular and therefore the Doppler correction is quite small ($<$2~ms) along the majority of the survey (Figure~\ref{fig:one_sta_real_survey}c).

