The last two decades have seen a sea change in the longevity, distribution, and sophistication of temporary ocean bottom seismic installations. The proliferation of ocean bottom seismometer (OBS) deployments has opened up new possibilities for understanding the ocean basins \citep[e.g.][]{Lin2016,Takeo2016}, continental margins \citep[e.g.][]{Janiszewski2015,Hawley2016,Lynner2017,Eilon2017}, and even inland submerged environments \citep[e.g.][]{Accardo2017}. 

However, even straightforward OBS installations present several unique challenges. Foremost among these is the inability to directly measure the location of the sensor at the seafloor. Precise knowledge of station location is essential for almost all seismological analysis. While the location of the ship is known at the time of deployment, as OBS instruments sink they may drift up to hundreds of meters from this point due to ocean currents and a non-streamlined basal profile. 

For broadband OBS deployments, it has long been accepted practice to conduct an acoustic survey in order to triangulate the position of the instrument \citep[e.g.,][]{Creager1982}. To accomplish this, ships send non-directional acoustic pulses (``pings'') into the water column. These are received by the OBS transponder which sends its own acoustic pulse in response. The time elapsed between the ship sending and receiving acoustic pulses is proportional to distance, which (for known ship location) may be used to locate the instrument. It is common for this analysis to be conducted by technicians at OBS instrument centers and provided latterly to PIs and data centers as station metadata. Some codes are proprietary intellectual property of the instrument centers, and others are available for a license fee. 

However, standard station location algorithms to date are lacking in certain respects. Water sound speed and even water depth are often assumed \textit{a priori}. Commonly, no corrections are made to account for the movement of the ship between sending and receiving acoustic signals, the horizontal offset between GPS and transponder location, or ray bending due to refraction through the water column. Robust uncertainty analysis, which would allow practitioners to gauge potential location errors, is either not conducted or communicated.

We present an OBS locator software for use by the marine geophysics community that can account for ship velocity, GPS-transponder offset, and ray bending. Our efficient inversion algorithm provides station location in three dimensions and solves for depth-averaged water sound speed. We use statistical tools to provide robust uncertainties on the instrument location as well as water velocity. The code is available in both MATLAB and Python to promote accessibility (see Data and Resources). In this article, we present the theory behind our algorithm, validate the inversion using synthetic testing, compare its accuracy with a previous tool, and carefully test a variety of survey patterns identifying optimal geometries for accurately recovering all model parameters, including the trade-off between depth and water velocity. Finally, we demonstrate its utility with real data from the 2018 Young Pacific ORCA (OBS Research into Convecting Asthenosphere) Experiment \citep{Gaherty2018}, revealing a network-wide clockwise-rotation that resembles a cyclonic mesoscale gyre. This study represents a first open-source tool for accurately locating instruments on the seafloor as well as a thorough investigation of survey geometries that will serve to inform future OBS deployments.