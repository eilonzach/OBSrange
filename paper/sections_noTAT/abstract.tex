\begin{abstract}
As the marine geophysics community continues to instrument the seafloor, data quality and instrument recoverability rely on accurate estimates of instrument locations on the ocean floor. However, freely available software for this estimation does not currently exist. We present \textit{OBSrange}, an open-source tool for robustly locating ocean bottom seismometers (OBS) on the seafloor. Available in both MATLAB and Python, the algorithm inverts two-way acoustic ranging travel-time data for instrument location, depth, and average water velocity. The tool provides comprehensive estimates of model parameter uncertainty including bootstrap uncertainties for all four parameters as well as an F-test grid search providing a 3D confidence ellipsoid around each station. We validate the tool using a synthetic travel-time dataset and find average horizontal location errors on the order of $\sim$4~m. An exploration of survey geometries shows that the so-called \textit{``PACMAN''} style survey pattern of radius 1~Nm with long ship-tracks towards and away from the instrument is optimal for resolving all parameters, including the trade off between instrument depth and water velocity. A survey radius of 0.75~Nm is sufficient for accurate horizontal locations (to within $\sim$5~m) with diminishing improvement as radius is increased. Depth and water velocity trade off perfectly for \textit{Circle} surveys, and \textit{Line} surveys are unable to resolve the instrument location orthogonal to the survey line; if possible, both geometries should be avoided. We apply our tool to the 2018 Young Pacific ORCA deployment in the south Pacific producing an average RMS data misfit of 1.96~ms with an average instrument drift of $\sim$170~m. Observed drifts reveal a clockwise-rotation pattern of $\sim$500~km diameter that resembles a cyclonic mesoscale gyre observed in the geostrophic flow field, suggesting a potential use for accurate instrument drifts as a novel proxy for depth-integrated flow through the water column.
\end{abstract}