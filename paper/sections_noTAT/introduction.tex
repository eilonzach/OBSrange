The last two decades have seen a sea change in the longevity, distribution, and sophistication of temporary ocean bottom seismic installations. The proliferation of ocean bottom seismometer (OBS) deployments has opened up new possibilities for understanding the ocean basins, continental margins, and even inland submerged environments. 

However, even straightforward OBS installations present several unique challenges. Foremost among these is the inability to directly measure the location of the sensor at the seafloor. Precise knowledge of station location is essential for almost all seismological analysis. While the location of the ship known at the time of deployment, OBS instruments are found to drift by up to hundreds of meters from this point due to water currents and a non-streamlined  basal profile. 

For broadband OBS deployments, it has long been accepted practice to conduct an acoustic survey in order to triangulate the position of the instrument. To accomplish this, ships send non-directional acoustic pulses into the water column. These are received by the OBS transponder which sends its own acoustic pulse in response. The time elapsed between the ship sending and receiving acoustic pulses is proportional to distance, which (for known ship location) may be used to locate the instrument. It is common for this analysis to be conducted by technicians at OBS instrument centers and provided latterly to PIs and data centers as station metadata. Some codes are proprietary intellectual property of the instrument centers, and others are available for a license fee. 

However, standard station location algorithms to date are lacking in certain respects. Water sound speed and even water depth are often assumed \textit{a priori}. Commonly, no correction is made for the movement of the ship. Robust uncertainty analysis, which would allow practitioners to gauge potential location errors, is either not conducted or communicated. 

We present an open-source OBS locator software for use by the marine geophysical community. Our efficient inversion algorithm provides station location in three dimensions and solves for depth-averaged water sound speed. We use statistical tools to provide robust uncertainties on the station location. We have made the code available in both MATLAB and Python to promote accessibility. In this article we present the theory behind our algorithm, validate the inversion using synthetic testing, demonstrate its utility with real data, and analyze a variety of location survey patterns so as to inform the planning of future OBS deployments. 