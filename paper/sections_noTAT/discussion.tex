An open-source tool for the remote location of instruments on the seafloor is warranted, given the growing number of ocean bottom deployments. At present, such a tool does not currently exist for the marine geophysics community. Our tool for precisely locating OBS on the seafloor is available in both MATLAB and Python for open use by the community. The utility of \textit{OBSrange} is demonstrated using both synthetic and real datasets. One of the main advantages of the algorithm that sets it apart from others is its ability to provide multiple estimates of model parameter uncertainty (bootstrap analysis and F-test confidence). The F-test grid search, in particular, provides confidence intervals in $x$-$y$ that will help inform recovery cruise efforts.

The \textit{PACMAN} survey geometry with a radius of $\sim$1~Nm is optimal for accurately recovering model parameters in the synthetic tests (Figure~\ref{fig:survey_geom_explore}), including the depth-water velocity trade-off. Typical horizontal locations errors for such a configuration are on the order of $\sim$4~m. A radius of 0.75~Nm is sufficient for accurate horizontal location (to within $\sim$5~m) but with increased RMS error in depth and water velocity. However, the smaller 0.75~Nm radius survey reduces the total survey duration by $\sim$25\% compared to the 1~Nm survey ($\sim$38~min compared to $\sim$50~min for an average ship velocity of 8~kn). If depth/velocity estimates are of lesser importance and/or time is limited, a smaller survey size of $\sim$0.75~Nm radius may be desirable. A radius larger than 1~Nm is likely not warranted, requiring more ship time at each site for little improvement in misfit. Additionally, failed acoustic returns are more likely to occur at greater distances from the instrument, resulting in data gaps. The radial legs of the survey where the ship travels toward and away from the instrument are crucial for resolving the depth-velocity trade-off. For this reason, the \textit{Circle} configuration cannot independently resolve depth and water velocity and should be avoided.

The \textit{Line} geometry warrants additional discussion as it is commonly used for locating OBS during active-source experiments because it is often the fastest method. However, the instrument location perpendicular to the line cannot be resolved. This is evident from the resolution matrix as well as the synthetic bootstrap tests. However, parallel to the line the instrument location is resolved quite well (to within $\sim$4~m). The instrument depth is also poorly resolved with RMS of $\sim$200~m. In order to resolve both horizontal dimensions, an alternative survey geometry with a range of ship-track azimuths should be used.

Observations of instrument drift from seasurface to seafloor are byproducts of the location algorithm if instrument drop points are precisely recorded. Figure~\ref{fig:meso_eddy} highlights both the precision of the \textit{OBSrange} algorithm as well as the potential for using instrument drift as an oceanographic observation. A clockwise rotation pattern is observed in instrument drift across the Young Pacific ORCA network that is consistent with a large cyclonic mesoscale feature, providing novel point measurements of depth-integrated flow through the water column that could be used to calibrate models of the vertical shear (Ryan Abernathey, \textit{pers. comm.}). With the further proliferation of seafloor data providing broader spatial and temporal sampling, measurements such as these could be used to estimate vertical structure of the water column. Furthermore, the network-wide depth-averaged water velocity is $\sim$1505~m, consistent with the average for this region expected in the month of April ($\sim$1509~m/s) [2009 NOAA World Ocean Atlas] and providing further confidence in the ability to accurately estimate average water velocity.

We find that the Doppler travel-time corrections only slightly improve RMS misfit for the synthetic tests (Figure~\ref{fig:compare_tool}) and do not improve misfit for the real data. One possible reason why the corrections fail to improve the misfit for real data may simply be the inability to accurately estimate ship velocity resulting from poor GPS spatial precision and/or poor spatial-temporal sampling along the ship tracks, especially when large data gaps are present. Additionally, the algorithm does not include a travel-time correction to account for a possible offset in the GPS receiver and the acoustic transponder relative to the instrument (i.e. it is assumed that they are colocated). Let us consider a worst-case scenario where the GPS and transponder are at opposite ends of the ship and one is closer to the instrument by $\sim$30~m. For a 1~Nm radius survey with the instrument at 5~km depth, the travel-time difference due to the separation would be $\sim$14~ms. However, for quasi-circular geometries such as \textit{PACMAN}, this timing error will be static around the perimeter of the circle effecting primarily the depth and water velocity; Thus, it should not significantly effect the estimated horizontal instrument location.