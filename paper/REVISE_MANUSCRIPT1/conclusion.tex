We present \textit{OBSrange}, a new open-source tool for robustly locating OBS on the seafloor. Two-way travel times between the ship and OBS are inverted for horizontal instrument position, instrument depth and depth-averaged water velocity. Uncertainties are calculated for all four parameters using bootstrap resampling, and an F-test grid search provides a 3D confidence ellipsoid around the station. The tool is validated using a synthetic travel-time dataset yielding typical horizontal location errors on the order of $\sim$4~m. Various survey geometries are explored through synthetic tests, and we find that the \textit{PACMAN} survey configuration with $\sim$1~Nm radius is the optimal geometry for robustly recovering the true instrument position while minimizing the trade-off between depth and water velocity. The \textit{Circle} configuration is unable to resolve depth and water velocity and should be avoided. The \textit{Line} survey pattern, commonly used in short-period OBS deployments, recovers instrument location parallel to the line but has no resolution in the orthogonal direction. If instrument depth and/or water velocity are of particular importance, a survey pattern such as \textit{PACMAN} is desirable, which contains long ship tracks toward and away from the instrument. If depth and water velocity are of lesser importance and/or time is restricted, a \textit{PACMAN} survey of radius $\sim$0.75~Nm is sufficient for resolving horizontal position to $\sim$5~m. The tool is applied to the 2018 \textit{Young Pacific ORCA} deployment yielding an average RMS data misfit of 1.96~ms and revealing a clockwise-rotation pattern in the instrument drifts with a diameter of $\sim$500~km that correlates with a cyclonic mesoscale feature. This observation further demonstrates the precision of \textit{OBSrange} and suggests the possibility of utilizing instrument drift data as an oceanographic tool for estimating depth-integrated flow through the water column.