A reliable tool for accurately locating instruments on the seafloor is paramount, given the growing number of ocean bottom deployments. We present the first such open-source OBS locator code that is freely available to the scientific community. One of the primary features of the tool is its ability to provide robust confidence bounds on the 3D instrument position on the seafloor, which will inform recovery cruise efforts as well as provide accurate station metadata which essentially all seismic analyses rely on. Furthermore, this article represents the first systematic exploration of survey geometries that we are aware of, which will help streamline future OBS deployments.

The \textit{PACMAN} survey geometry with a radius of $\sim$1~Nm is optimal for accurately recovering all model parameters in the synthetic tests (Figure~\ref{fig:survey_geom_explore}), including the depth-water velocity trade-off. Typical horizontal locations errors for such a configuration are on the order of $\sim$4~m. A radius of 0.75~Nm is sufficient for accurate horizontal location (to within $\sim$5~m) but with increased RMS error in instrument depth and water velocity. However, the smaller 0.75~Nm radius survey reduces the total survey duration by $\sim$25\% compared to the 1~Nm survey ($\sim$38~min compared to $\sim$50~min for an average ship velocity of 8~kn). If depth/velocity estimates are of lesser importance and/or time is limited, the smaller 0.75~Nm radius may be desirable. A survey radius larger than 1~Nm is likely not warranted, requiring more ship time at each site for little improvement in misfit. Additionally, failed acoustic returns are more likely to occur at greater distances from the instrument, resulting in data gaps which will negatively impact the inversion. Some ship captains prefer only to steam along straight lines; in such cases, the \textit{Diamond} survey with 1~Nm radius is a viable alternative, given its comparable performance to the \textit{PACMAN} geometry (Figure~\ref{fig:survey_geom_explore}a--c). The radial legs of the survey where the ship travels toward and away from the instrument are crucial for resolving the depth-velocity trade-off. For this reason, the \textit{Circle} configuration cannot independently resolve depth and water velocity and should be avoided.

The \textit{Line} geometry warrants additional discussion as it is commonly used for locating OBS during active-source experiments because it is often the simplest pattern. However, the instrument location perpendicular to the line cannot be resolved. This is evident from the resolution matrix as well as the synthetic bootstrap tests. Parallel to the line, the instrument location is resolved quite well (to within $\sim$4~m). The instrument depth is also poorly resolved with RMS of $\sim$200~m. In order to resolve both horizontal dimensions and depth, an alternative survey geometry with a range of ship-track azimuths (or even two perpendicular lines crossing the instrument, such as the \textit{Cross} geometry) may be used.

Observations of instrument drift from seasurface to seafloor are byproducts of the location algorithm if instrument drop points are precisely recorded. Figure~\ref{fig:meso_eddy} highlights both the precision of the \textit{OBSrange} algorithm as well as the potential for using instrument drift as an oceanographic observation. A clockwise rotation pattern in instrument drift is observed across the Young Pacific ORCA network that is consistent with a large cyclonic mesoscale feature, providing novel point measurements of depth-integrated flow through the water column that could be used to calibrate models of the vertical shear (Ryan Abernathey, \textit{pers. comm.}). With the further proliferation of seafloor data providing broader spatial and temporal sampling, measurements such as these could be used to estimate vertical structure of the water column. The network-wide depth-averaged water velocity is $\sim$1505~m/s with standard deviation $\sim$4.5~m/s, consistent with the regional decadal average for the month of April ($\sim$1509~m/s) from the 2009 World Ocean Database (see Data and Resources).

We find that the Doppler travel-time corrections improve RMS travel-time misfit by only $\sim$0.5~ms ($\sim$11\% reduction) for the synthetic test (Figure~\ref{fig:compare_tool}) and do not improve RMS misfit for the real data. However, the test shows a reduction in horizontal errors of $\sim$2.5~m ($\sim$40\%) when using the correction, and therefore, we include the Doppler correction as an option in the code. One possible reason why the corrections fail to improve the travel-time misfit for real data may simply be the inability to accurately estimate ship velocity resulting from poor GPS spatial precision and/or poor spatial-temporal sampling along the ship tracks, especially when large data gaps are present. Additionally, the algorithm does not include a travel-time correction to account for a possible offset in the GPS receiver and the acoustic transponder relative to the instrument (i.e. it is assumed that they are colocated). Let us consider a worst-case scenario where the GPS and transponder are at opposite ends of the ship and one is closer to the instrument by $\sim$30~m. For a 1~Nm radius survey with the instrument at 5~km depth, the travel-time difference due to the separation would be $\sim$14~ms. However, for quasi-circular geometries such as \textit{PACMAN}, this timing error will be static around the perimeter of the circle effecting primarily the depth and water velocity; Thus, it should not significantly effect the estimated horizontal instrument location.