We present \textit{OBSrange}, a new open-source tool for robustly locating OBS on the seafloor. Two-way travel times between the ship and OBS are inverted for horizontal instrument position, instrument depth, depth-averaged water velocity, and turn-around time. Bootstrap uncertainties are calculated for all five parameters and F-test confidence bounds for the three spatial dimensions. The tool is validated using a synthetic travel-time dataset yielding typical horizontal location errors on the order of $\sim$4~m. Various survey geometries are explored through synthetic bootstrap tests, and we find that a \textit{PACMAN} style survey configuration with $\sim$1~Nm radius is the optimal geometry for robustly recovering the true instrument position while minimizing the trade off between depth and water velocity. If instrument depth and/or water velocity are of particular importance, a survey pattern which contains long ship tracks toward and away from the instrument is desirable. The \textit{Circle} configuration is unable to resolve depth and water velocity and should be avoided. The \textit{Line} survey pattern, commonly used in short-period OBS deployments lacks resolution orthogonal to the line. The tool is applied to the 2018 \textit{Young Pacific ORCA} deployment with an average RMS data misfit of 1.96~ms, revealing a clockwise rotation pattern with a diameter of $\sim$500~km that correlates with a cyclonic mesoscale feature that occurred during the deployment. This observation further demonstrates the precision of the tool and suggests a possibility for using instrument drift data for oceanographic purposes.

%- provide uncertainties
%- PACMAN geometry is best
%- Circle is worst
%- Line test can't resolve orthogonal direction
%- Turn-around time is not well resolved

%The circle geometry

%To demonstrate its utility,  is applied to synthetic and real datasets, compared with a previous tool, and used to investigate optimal survey geometries for accurate model parameter recovery. Bootstrap and F-test uncertainties are reported, providing exceptional measures of uncertainty. The tool is validated using a synthetic dataset 