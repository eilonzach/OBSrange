

\subsection{The forward problem}

We wish to locate an instrument which rests at unknown position and depth on the ocean floor. Taking the drop coordinates as the center of a Cartesian coordinate system in which $x$ is positive towards East, $y$ is positive towards North, and $z$ is positive upwards from the sea surface, the instrument lies at location $(x_O,y_O,z_O)$. The time taken for an acoustic pulse to travel from the ship to the instrument and back is a function of the sound speed in water ($V_P$), and the location of the ship, as well as the ``turn-around time'' ($\tau$) that corresponds to the (fixed) processing time between the OBS transducer receiving a ping and sending its response. In detail, we must account for the possibility that if the ship is under way, its position changes between sending and receiving pings. Thus, the total travel time, $T$, is:
\begin{equation}
T = \frac{r_s + r_r}{V_P} + \tau \label{eq:forward_send_receive}
\end{equation}
where
\begin{align}
	r_s &= \sqrt{(x_s - x_O)^2 + (y_s - y_O)^2 + z_O^2}\\
	r_r &= \sqrt{(x_r - x_O)^2 + (y_r - y_O)^2 + z_O^2}
\end{align}

where subscript ``$s$'' indicates the ship sending a ping and ``$r$'' indicates the ship receiving the OBS's response. These positions are related by the velocity ($\mathbf{u} = (u_x,u_y,0)$) of the ship:

\begin{equation}
\begin{pmatrix} x_s\\y_s\\0 \end{pmatrix} = \begin{pmatrix} x_r\\y_r\\0 \end{pmatrix} - T\begin{pmatrix} u_x\\u_y\\0 \end{pmatrix}
\end{equation}

It follows that, to a close approximation,

\begin{align*}
r_s &\approx r_r - T \left(\mathbf{u} \cdot \mathbf{\hat{r}}_r \right)\\
	&= r_r - \delta r
\end{align*}

where $\mathbf{\hat{r}}_r$ is the unit-vector pointing from the instrument to the ship at the time of receiving. If we know the  distance $\delta r$ we can account for the send-receive timing offset related to a change in ship's position, by computing a correction time, $\delta T = \delta r/V_P$. Substituting this into equation (\ref{eq:forward_send_receive}), we have

\begin{equation}
T + \delta T = \frac{2 r_r}{V_P} + \tau \label{eq:forward}
\end{equation}

\subsection{The inverse problem}

If travel times are known between the OBS and certain locations, but the position of the OBS is not, equation (\ref{eq:forward}) can be thought of as a non-linear inverse problem, of the form $ \mathbf{d} = g(\mathbf{m})$, where $g(\mathbf{m})$ represents the forward-model. The model contains five parameters: $\mathbf{m} = \{x_O,y_O,z_O,V_P,\tau\}$. The data, $\mathbf{d}$, are a vector of corrected travel times, $T+\delta T$ (note that $\delta T$ is itself a function of $\mathbf{m}$; this will be adjusted iteratively). This type of problem can be solved iteratively using Newton's method \citep{William:2012vh,Menke:1984vh}:

\begin{equation}
	\mathbf{m}_{k+1} = \mathbf{m}_k + \left[\mathbf{G}^{\text{T}} \mathbf{G}\right]^{-1} \mathbf{G}^{\text{T}} \left(\mathbf{d} - g(\mathbf{m}_k)\right)
\end{equation}

where $\mathbf{G}$ is a matrix of partial derivatives: $G_{ij} = \partial d_i/\partial m_j$, as follows:

\begin{align*}
\frac{\partial d_i}{\partial x_O} &= 
	-\frac{2 x_O}{V_P} \left( (x_i - x_O)^2 + (y_i - y_O)^2 + z_O^2 \right)^{-\frac{1}{2}}\\
\frac{\partial d_i}{\partial y_O} &= 
	-\frac{2 y_O}{V_P} \left( (x_i - x_O)^2 + (y_i - y_O)^2 + z_O^2 \right)^{-\frac{1}{2}}\\
\frac{\partial d_i}{\partial z_O} &= 
	\frac{2 z_O}{V_P} \left( (x_i - x_O)^2 + (y_i - y_O)^2 + z_O^2 \right)^{-\frac{1}{2}}\\	
\frac{\partial d_i}{\partial V_P} &= 
	-\frac{2}{V_P^2} \left( (x_i - x_O)^2 + (y_i - y_O)^2 + z_O^2 \right)^{\frac{1}{2}}\\	
\frac{\partial d_i}{\partial \tau} &= 1\\	
\end{align*}

We use the drop coordinates and water depth (if available from multibeam) as a starting model, along with $V_P = 1500$ m/s and $\tau =$10 ms. If we consider the setup of the problem, there is some degree of trade off between the water depth and the water velocity. Simplistically, if all survey measurements are made at a constant distance from the station (\textit{e.g.}, if the survey is a circle centered on the station) then these parameters co-vary perfectly. As a result, the inverse problem is ill-posed and, like all mixed-determined problems, requires regularization. We use constraint equations to damp perturbations in $V_P$, which is not likely to vary substantially from 1500 m/s, and $\tau$, which should not vary substantially from $\sim$13 ms (Ernest Aaron, \textit{pers. comm.}):

\begin{equation}
	\mathbf{F} = 
	\left[\begin{matrix}
	\mathbf{G}\\\mathbf{H}
	\end{matrix}\right] 
	\hspace{15mm}
	\mathbf{f} = 
	\left[\begin{matrix}
	\mathbf{d} - g(\mathbf{m})\\\mathbf{0}
	\end{matrix}\right] 
\end{equation}
where
\begin{equation}
	\mathbf{H} = 
	\left(\begin{matrix}
	0 &&&&\\ &0&&&\\ &&0&&\\ &&&\gamma_{V_P}&\\ &&&& \gamma_{\tau}
	\end{matrix}\right) 
\end{equation}

We have had success using $\gamma_{V_P} = 5\times10^{-8}$ and $\gamma_{\tau} = 0.2$.  Finally, we implement global norm damping to stabilize the inversion, through parameter $\epsilon = 10^{-10}$, such that the equation to be solved becomes:

\begin{equation}
	\mathbf{m}_{k+1} = \mathbf{m}_k + \left[ \mathbf{F}^{\text{T}} \mathbf{F} + \epsilon\mathbf{I} \right]^{-1} \mathbf{F}^{\text{T}} \mathbf{f} \label{eq:inverse}
\end{equation}

This equation is solved iteratively, until the root-mean-squared (RMS) of the misfit $(T+\delta T-g(\mathbf{m})|)$ decreases by less than 0.1 ms compared to the previous iteration. This criterion is usually reached after $\sim$4 iterations. 

\subsection{Errors and uncertainty}
In order to estimate the uncertainty in our model we bootstrap survey timing data with a balanced resampling approach. In each iteration the algorithm inverts a random sub-sample of the true data set, with the constraint that all data points are eventually sampled an equal number of times. This approach provides empirical probability distributions of possible model parameters, but does not straightforwardly offer quantitative estimates of model uncertainty because the goodness of data fit for each run in the bootstrap iteration is ignored (that is, within each iteration, a model is found that best fits the randomly sub-sampled dataset, but in the context of the full dataset, the fit and uncertainty of that particular model may be very poor). For more statistically robust uncertainty estimates, we perform a grid search over ($x_O,y_O,z_O$) within a region centred on the bootstrapped mean location, 
$(x_{O_{\text{best}}},y_{O_{\text{best}}},z_{O_{\text{best}}})$. For each perturbed location, ($x_O^{\prime},y_O^{\prime},z_O^{\prime}$), we use an F-test to compare the norm of the data prediction error to the minimum error, assuming they each have a $\chi^2$ distribution. The effective number of degrees of freedom, $\nu$ is calculated as 
\begin{equation}
\nu = N_f - \text{tr}(\mathbf{F}\mathbf{F}_{\text{inv}})
\end{equation}
where $N_f$ is the length of vector $\mathbf{f}$ and $\text{tr}()$ denotes the trace. Using the F-test, we can evaluate the statistical probability of the true OBS location departing from our best-fitting location by a given value. 

Some care is required in implementing this grid search. Since $z_O$ covaries with $V_P$, varying $z_O$ quickly leads to large errors in data prediction as $|z_O^{\prime}-z_{O_{\text{best}}}|$ increases if one holds $V_P$ fixed. As a result, it appears as if the gradient in the error surface is very sharp in the $z$ direction, implying this parameter is very well resolved; in fact, the opposite is true. We find the empirical covariance of $z_O$, $V_P$, and $\tau$ by performing principal component analysis on the bootstrap model solutions. We then use the largest eigenvector to project perturbations in $z_O$ within the grid search onto the other two parameters, adjusting them appropriately as we progress through the grid search. 







