\begin{abstract}
As the marine geophysics community continues to instrument the seafloor, data quality and easy instrument recoverability rely on accurate estimations of the instrument locations on the ocean floor. However, freely-available software for this estimation does not currently exist. We present \textit{OBSrange}, an open-source tool for robustly locating ocean bottom seismometers (OBS) on the seafloor. Available in both MATLAB and Python, the algorithm inverts acoustic ranging two-way travel time data for instrument position, depth, average water velocity, and turn-around time. The tool provides comprehensive estimates of model parameter uncertainty including bootstrap uncertainties for all five parameters as well as an F-test grid search that provides a 3D confidence ellipsoid around each station. We validate the tool using a synthetic travel-time dataset and find average horizontal location errors on the order of $\sim$4~m. An exploration of survey geometries shows that the so-called \textit{``PACMAN''} style survey pattern of radius $\geq$1~Nm with long ship-tracks towards and away from the instrument is optimal for resolving all parameters, including the trade-off between depth and water velocity. We apply our tool to the 2018 \textit{Young Pacific ORCA} deployment in the south Pacific and observe an average RMS data misfit of 1.96~ms. We observe instrument drifts of 170~m on average that form a clockwise-rotation pattern of $\sim$500~km diameter that coincides with a cyclonic mesoscale feature, suggesting a potential use for instrument drifts as a proxy for depth-integrated flow through the water column.
\end{abstract}