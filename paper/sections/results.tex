
\subsection{Demonstration on synthetic data}
We validated our algorithm by checking that it correctly recovers the (known) location of synthetic test stations. Synthetic two-way travel times were computed by interpolating the ship's position within a fixed survey pattern at one-minute intervals, sending straight-line rays to the instrument and back, and adding the turn-around time. This travel time includes the change in ship's position between sending and receiving; since the position of the ship at the time it receives the acoustic pulse is itself dependent on the travel time, we iterated on this value until the time and position converged to give an error of \mbox{$<10^{-6}$ s}. Only the location and absolute time at the time the ship receives the acoustic pulse was recorded for the inversion, mimicking the data obtained from the EdgeTech deck box. We then added Gaussian random noise to the resultant travel times using a standard deviation of 4 ms, to account for measurement noise, errors in ship GPS location, and local changes in water velocity. Lastly, we randomly dropped out $\sim$20\% of the travel time data points, simulating the occasional null return from the acoustic survey. This testing procedure was designed to mimic the idiosyncrasies of real acoustic surveys as closely as possible. 

Figure \ref{fig:one_sta_synth} shows the result of an inversion at a single station. For this inversion, we included a correction for a Doppler shift introduced by the ship's motion, estimating ship velocity from the timing and location of survey points. The inversion was successful in locating the OBS station: the estimated location is 3.02 m from the the true location (Figure \ref{fig:one_sta_synth}). This misfit is extremely small in the context of $\sim$320 m of drift, a survey radius of $\sim$3700 m, and a water depth of $\sim$5300 m. Moreover, the true location falls well within the uncertainty bounds estimated from the F-test and the bootstrap analysis. 

In order to obtain statistics on the general quality of the synthetic recovery, we performed this test for 10,000 synthetic OBS stations, as follows: For each iteration, a synthetic station location was determined relative to a fixed drop point by drawing x- and y- drifts from zero-centered Gaussian distributions with standard deviations of 100 m (only in rare cases are stations thought to drift further than $\sim$200 m). The depth perturbation, turn-around time, and water velocity perturbation were similarly randomly selected, with mean values of 5000 m, 13 m/s, and 1500 m/s and standard deviations of 50 m, 3 ms, and 10 m/s, respectively.  For tests of the basic location algorithm, we held the survey geometry constant, using the \textit{PACMAN} configuration with a radius of 1 Nm (Section \ref{sec:surv_geom_tests}). 

%Unlike for the real data, we did not bootstrap the individual synthetic locations and take a median, but inverted each station once using Newton's method (Equation \ref{eq:inverse}).   
The results of these tests show that on average our inversion is highly successful in correctly locating the OBS stations. The mean location errors in the x-, y-, and z- directions were 0.038 m, 0.152 m, and -0.599 m respectively, demonstrating there was no systematic bias in the locations. The mean errors in water velocity and turn-around time were indistinguishable from zero, showing that estimation of these parameters was also not biased. The mean absolute horizontal location error was 2.31 m, with a standard deviation of 1.22 m. 95\% of the absolute horizontal station location errors were less than 4.58 m. There was no relationship observed between station drift (\textit{i.e.}, the distance between the synthetic OBS station and the drop point) and the location error, indicating that as long as stations settle within the survey bounds they will be well located. A corollary to this observation is that location estimates should not be biased by incorrectly recorded drop locations. 

We observed a strong trade-off between water velocity and depth, which was responsible for the somewhat larger standard error in station depth estimates, which was \mbox{9.6 m}. This uncertainty is likely of negligible concern for most OBS practitioners, but if precise depths are important then a survey geometry that includes more tracks towards and away from the station would be preferable (in addition to verification using acoustic echo-sounders that implement precise water-velocity profiles from XBT data).

\subsection{Application to PacificArray deployment}
We applied the location algorithm to acoustic surveys carried out during the \textit{Young Pacific ORCA (OBS Research into Convecting Asthenosphere)} deployment in the central Pacific ocean during April and May of 2018. The OBS array comprised 30 SIO broadband instruments deployed from the R/V Kilo Moana in water depths of $\sim$4400-4800 m. Acoustic  surveys were carried out using an EdgeTech 8011M Acoustic Transceiver command and ranging unit, attached to a hull-mounted 12 kHz transducer. The relatively calm seas allowed for ideal survey geometry at almost all sites, with a ship speed of \mbox{$\le$8 knots} at a maximum radius of \mbox{$\sim$1.3 Nm}. 

An example station inversion, as well as the graphical outputs of the location software, is shown in Figures \ref{fig:one_sta_real_survey}-\ref{fig:one_sta_real_ftests}. Ship velocity is estimated from the survey data by differencing survey points. In theory, this could be used to correct doppler shifts (Figure \ref{fig:one_sta_real_survey}c) in travel time (as in the synthetic tests), but we found that this correction did not substantially improve data fit for real stations and so did not apply it to this data set, although it is included as an option in the location codes. The small RMS data misfit of $\sim$1.6 ms attests to the quality of the survey measurements and the appropriateness of our relatively simple location algorithm. The southwestwards drift of \mbox{$\sim340$ m} demonstrates that ocean currents can substantially displace the final OBS location from their surface drop point. 

The 30 stations in this array drifted an average distance of 170 m. The mean data RMS misfit was 1.96 ms and the estimated 95\% percentile location error based on the bootstrap analysis was 1.13 m. The water depth estimated by the inversion was systematically shallower than that measured using the shipboard multibeam instrument, differing by an average value of 18.6 m. Assuming the multibeam depths, which are computed using a water sound speed profile that is validated daily by XBT measurements, are correct, this discrepancy indicates that the inversion systematically overestimates sound speed slightly. 

Without accurate seafloor corroboration from an ROV, it is not possible to directly verify the locations of stations within the Pacific ORCA array. However, we  obtain indirect support for the success of the location algorithm by considering the drift of all stations within this array (Figure \ref{fig:meso_eddy}). Taken together, the direction and magnitude of drift depicts a pattern of clockwise rotation with a minimum diameter of $\sim$500 km. This system appears to correlate with a meso-scale ocean gyre, with a direction, location, and approximate size that is consonant with large-scale patterns of geostrophic flow modeled in this location within the time frame of our deployment. We speculate that a large fraction of the OBSs' lateral drift is achieved in the upper \textit{\textbf{X}} m of the ocean, where horizontal flow amplitudes are highest [\#REF]. The fact that we are able to discern this pattern from our estimated locations is a testament to the accuracy of the OBSrange algorithm. This observation also raises the intriguing possibility of using OBS instruments as ad hoc depth-integrated flow meters for the oceans. 

\subsection{Comparison to previous tools}

\begin{itemize}
\item Re-run synthetics with damped Vw, etc.\\
\item Talk about null points\\
\item Show difference in error between our method and theirs\\
\item latlon2xy conversion\\
\end{itemize}

We compared our location algorithm with a tool developed by engineers at Scripps Institution of Oceanography (SIO) that has previously been used to locate OBS on the seafloor. This unpublished tool, hereby referred to as \textit{SIOgs}, performs a grid search in $x$--$y$ holding $z$ fixed at the reported drop-point depth and assumes a water velocity of 1500 m/s and turn-around time of 13 ms. The grid search begins with grid cells of 100$\times$100~m and iteratively reduces their size to 0.1$\times$0.1~m. In contrast to our algorithm, \textit{SIOgs} does not account for: 1) the $\delta T$ (Doppler) correction due to the changing ship position between sending and receiving, 2) the ellipsoidal shape of the earth when converting between latitude--longitude and $x$--$y$, 3) variations in $z$, $\tau$, and $V_p$, and 4) automated identification and removal of low-quality travel-time data. 

To quantitatively compare our algorithm with \textit{SIOgs}, as well as the importance of the 4 additional features that our algorithm includes, we performed 9 separate inversions of a synthetic dataset for a \textit{PACMAN} survey geometry with 1 Nm radius (Figure 5). The inversions include the complete \textit{OBSrange} algorithm as well as several variants where features have been damped or removed to assess their importance; details of the inversions are given in Table 1. Our algorithm locates the station to within $\sim$1.5~m of the true location, while \textit{SIOgs} locates it $\sim$42~m from the true position, far beyond the 95\% F-test contour (Figure 5a). 

The grid search method is very susceptible to mislocations due to anomalous data. Inversion \textit{SIOgs no QC} in Figure 5 includes a single anomalous travel-time measurement 2000~ms away its true value, causing the station to be mislocated by $\sim$130 m. Although such outliers can be manually removed, they may easily be overlooked and therefore we apply a quality control step based on travel-time residuals of the starting location that removes anomalous residuals with magnitudes $>$500 ms.

In addition to showing the full potential of \textit{OBSrange}, we demonstrate the importance of including the "Doppler" correction ($\delta T$ in equation (5)), accounting for Earth's ellipsoidal shape when converting latitude and longitude to $x$--$y$.





\subsection{Exploration of survey pattern geometries} \label{sec:surv_geom_tests}