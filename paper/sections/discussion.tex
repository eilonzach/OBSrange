We introduce a new tool for precisely locating OBS instruments on the seafloor, available in both MATLAB and Python for open use by the scientific community. In this paper, we demonstrate the application of this tool to synthetic and real datasets as well as compare with previous tools and use it to investigate optimal survey geometries.

Figure~\ref{fig:meso_eddy} highlights the importance of accurate recording of OBS drop points. We are able to see a large meso-scale eddy.

The algorithm does not include a travel-time correction that accounts for differences in the position of the GPS and transponder on the ship (i.e. it is assumed that they are colocated). However, due to the symmetry of quasi-circular geometries such as \textit{PACMAN}, the signal should cancel [figure out how to better word this]. Furthermore, taking into account the nearly vertical ray paths of the incoming acoustic pulse 