An open-source tool for the remote location of instruments on the seafloor is commensurate with the growing number of ocean bottom deployments; however, such a tool does not currently exist for the marine geophysics community. We introduce a new tool for precisely locating OBS on the seafloor, available in both MATLAB and Python for open use by the community. The utility of \textit{OBSrange} is demonstrated using both synthetic and real datasets. One of the main advantages of the algorithm that sets it apart from others is its ability to provide multiple estimates of model parameter uncertainty (bootstrap analysis and F-test confidence), which are especially important for efficient instrument retrieval following a deployment. The F-test in particular, provides confidence intervals in $x$-$y$ that help indicate where instruments are likely to surface, informing recovery cruise efforts.

The \textit{PACMAN} survey geometry with a radius of $\sim$1~Nm is optimal for accurately recovering model parameters in the synthetic tests (Figure~\ref{fig:survey_geom_explore}). A smaller radius results in a stronger depth-velocity trade off as well as a decrease in horizontal resolution. A radius larger than 1~Nm yields slightly more accurate results, but with diminishing return. Larger surveys require more ship-time spent at each site, and perhaps more importantly, the acoustic transponder may lose communication at ranges beyond $\sim$1.25~Nm. The radial legs of the survey where the ship travels toward and away from the instrument are crucial for resolving the depth-velocity trade off. For this reason, the \textit{Circle} configuration cannot independently resolve depth and water velocity and therefore, it should be avoided.

Observations of instrument drift from seasurface to seafloor are byproducts of the location algorithm if instrument drop points are precisely recorded. Figure~\ref{fig:meso_eddy} highlights both the precision of the \textit{OBSrange} algorithm as well as the potential importance of instrument drift as an oceanographic observation. A clockwise rotation pattern is observed in instrument drift across the Young Pacific ORCA network that is consistent with a large cyclonic mesoscale feature, providing novel point measurements of depth-integrated flow through the water column that could be used to calibrate models of the vertical shear (Ryan Abernathey, \textit{pers. comm.}). With the further proliferation of seafloor data providing broader spatial and temporal sampling, measurements such as these could be used to estimate vertical structure of the water column. Furthermore, the network average depth-integrated water velocity is [number] consistent with the average for this region of 1509~m/s, taken from the 2009 NOAA World Ocean Atlas.

The \textit{Line} geometry warrants additional discussion as it is commonly used for locating OBS during active-source experiments. Perhaps unsurprisingly, the location of the instrument perpendicular to the line cannot be resolved. This is evident from the resolution matrix as well as the synthetic bootstrap tests. However, the instrument location is resolved quite well (to within $\sim$4~m) parallel to the line. The instrument depth is also poorly resolved with RMS of $\sim$200~m. An alternative survey geometry with orthogonal ship tracks, such as the \textit{Cross} patterns, is required to resolve both horizontal dimensions.

The so-called ``Doppler'' corrections slightly improve RMS misfit for the synthetic tests (Figure~\ref{fig:compare_tool}) but not for the real data. There are several possible reasons why the corrections fail to improve the misfit for real data. One may simply be the inability to accurately estimate ship velocity resulting from poor GPS spatial precision and/or poor spatial-temporal sampling along the ship tracks, especially when large data gaps are present. Additionally, the algorithm does not include a travel-time correction to account for a possible offset in the acoustic transmitter and receiver relative to the instrument (i.e. it is assumed that the receiver and transmitter are colocated). Let us consider a worst-case scenario where the transmitter and receiver are at opposite ends of the ship and one is closer to the instrument by $\sim$50~m. For a 1~Nm radius survey with the instrument at 5.5~km depth, the travel-time difference due to the separation is $\sim$11~ms. However, for quasi-circular geometries such as \textit{PACMAN}, this timing error will be static and mostly effect the turn-around time; it should not significantly effect the horizontal instrument location.

Our algorithm is not robust to significant deviations in the turn-around time from the value provided by the transponder manufacturer, regardless of the survey geometry used. The water velocity and depth trade off through the turn-around time (Figure~\ref{fig:resolution_correlation}), meaning that the three cannot be independently resolved. Thus, allowing $\tau$ to vary freely in the inversion leads to an instability where $V_P$ and $z$ reach unrealistic values. However, this is typically not an issue as the true turn-around time is not expected to deviate significantly from the value provided by the transponder manufacturer and therefore, can be heavily damped.