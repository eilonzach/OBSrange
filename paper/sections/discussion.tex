An open-source tool for the remote location of instruments on the seafloor is commensurate with the growing number of ocean bottom deployments; however, such a tool does not currently exist for the marine geophysics community. We introduce a new tool for precisely locating OBS on the seafloor, available in both MATLAB and Python for open use by the community. The utility of \textit{OBSrange} is demonstrated using both synthetic and real datasets. One of the main advantages of the algorithm is its ability to provide multiple estimates of model parameter uncertainty (bootstrap analysis and F-test confidence), which are especially important for efficient instrument retrieval following a deployment. The F-test in particular, provides confidence intervals in $x$-$y$ that help indicate where instruments are likely to surface, informing recovery cruise efforts.

Observations of instrument drift from seasurface to seafloor are byproducts of the location algorithm if instrument drop points are precisely recorded. Figure~\ref{fig:meso_eddy} highlights both the accuracy of the \textit{OBSrange} locations as well as the potential importance of instrument drift as an oceanographic observation. A clockwise rotation pattern is observed at the Young Pacific ORCA network that is consistent with a large cyclonic mesoscale feature, providing novel point measurements of depth-integrated flow through the water column that could be used to calibrate models of the vertical shear (Ryan Abernathey, \textit{pers. comm.}). With the further proliferation of seafloor data providing broader spatial and temporal sampling, measurements such as these could be used to estimate vertical structure of the water column. Furthermore, the network average depth-integrated water velocity is [number] consistent with the average for this region from the NOAA World Ocean Atlas of 1509~m/s.

We find that the \textit{PACMAN} survey geometry with a radius of $\sim$1~Nm is optimal for accurately recovering model parameters in the synthetic tests (Figure~\ref{fig:survey_geom_explore}). A smaller radius results in a stronger depth-velocity tradeoff as well as a decrease in horizontal precision. A radius larger than 1~Nm yields slightly more accurate results, but with diminishing return. Larger surveys require more ship-time spent at each site, and perhaps more importantly, the acoustic transponder may lose communication at ranges beyond $\sim$1.25~Nm.

The so-called ``Doppler'' corrections slightly improve RMS misfit for the synthetic tests (Figure~\ref{fig:compare_tool}) but not for the real data. There are several possible reasons why the corrections fail to improve the misfit for real data. One may simply be the inability to accurately estimate ship velocity resulting from poor GPS spatial precision and/or poor spatial-temporal sampling along the ship tracks, especially when large data gaps are present. Additionally, the algorithm does not include a travel-time correction that accounts for possible differences in the position of the GPS receiver and transponder on the ship (i.e. it is assumed that they are colocated) relative to the instrument. In a worst-case scenario, if the GPS and transponder are at opposite ends of the ship and one is closer to the instrument (say, $\sim$50~m apart) for a 1~Nm radius survey with the instrument at $\sim$5.5~km depth, the travel-time difference due to the separation is $\sim$11~ms. However, for quasi-circular geometries such as \textit{PACMAN}, this timing error will be static mostly effecting the turn-around time, and should not significantly effect the horizontal instrument location.

The turn-around time is not well resolved by our algorithm for any of the geometries tested (Figure~\ref{fig:survey_geom_explore}). The water velocity and depth trade off through the turn-around time, suggesting that the three cannot be independently solved for